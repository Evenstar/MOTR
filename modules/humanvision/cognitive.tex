\documentclass[12pt, reqno]{amsart}
\usepackage{tikz}
\usetikzlibrary{positioning}
\usepackage{listings}
\usepackage{color}
\usepackage{verbatim}
\usepackage{algorithmicx }
\usepackage{float}
\usepackage{algpseudocode}
\usepackage{amsmath,amssymb,amsfonts,amsthm}
%\addtolength{\oddsidemargin}{-1.in}
\usepackage{fullpage}
\title{Reading Notes on Cognitive Neuroscience}
\author{}
\date{}
\begin{document}
\maketitle
\section{Unit I: Memory}
\textbf{Hippocampal-perirhinal Theory} suggests that the hippocampus processes information relatively slowly and is associational and spatial, whereas the perirhinal cortex processes information more rapidly and is item based.  Neurons in the hippocampus signal information about spatial positions or associations between items(i.e., recollection) whereas neurons in the perirhinal cortex signal information about the novelty of individual items(i.t., familiarity). 
\textbf{Evidence} Single-cell recordings in experimental animals have shown the perirhinal neurons do indeed show a stronger response when an item is first presented than when the same item is shown again. 

\section{Working Memory}
Working memory has three phases: encoding, delay and response.  During encoding phase, one or more items of information is incorporated into working memory. During the delay phase, the encode information is maintained in working memory. Finally, during the response phase, an action is executed on the basis on the maintained information. \\
\textbf{Fact 1} Increases in delay-period activity have been associated with better working memory performance. \\
\textbf{Fact 2} Load effect.  Delay-period activity in dorsolateral prefrontal cortex was greater during maintenance of five faces than when the subject was asked to maintain three faces in working memory.\\

\noindent \textbf{Phonological working memory} The  phonological similarity effect refers tot he fact that working memory for letters is worse when the letters sound similar than when they sound different. The word length effect refers to the fact that people can hold more words in the working memory when the words are short than when the words are long. The phonological similarity effect suggests the existence of a store that maintains information in a phonological format. The word length effect suggests the existence of an articulatory rehearsal process and this process is slowed down by more complex words. The phonological store is linked to the left interior parietal cortex and the articulatory rehearsal process is linked to the posterior part of the left inferior front gyrus, which corresponds to Broca's area. The left inferior parietal cortex was activated for phonologically similar words but not for different words, and the same area can be activated for pseudowords. Broca's area showed greater activity for three-syllable than for one-syllable words. \\

\noindent \textbf{Grpahemic working memory} The left inferior temporal region showed load effects in the delay period activity for words, but not for pseudo-words. Similar to phonological working memory, but in left inferior temporal area.














\end{document}